\phantomsection
\addcontentsline{toc}{section}{ÖZET}


\begin{center}
\textbf{\large ÖZET}

\textbf{WEB TABANLI GEZİ UYGULAMASI \\ Gezgin Haritası}
\end{center}

\textbf{Projenin Amacı:} Bu projenin amacı, Türkiye'nin farklı şehirlerini gezmek isteyen kullanıcılara kolay, erişilebilir ve etkileşimli bir dijital rehber sunmaktır. Harita tabanlı bir arayüz aracılığıyla kullanıcılar ilgilendikleri şehri seçerek o bölgeye ait gezilecek yerleri, ilçe detaylarını ve rota planlama özelliklerini görüntüleyebilmektedir. Böylece kullanıcılar tek bir platform üzerinden hem seyahat planlamalarını yapabilmekte hem de şehirler hakkında kapsamlı bilgiye hızlıca ulaşabilmektedir.

\textbf{Projenin Kapsamı:} Bu proje, kullanıcıların Türkiye'nin farklı illerini keşfetmelerini sağlayacak bir web uygulamasının geliştirilmesini kapsamaktadır. Frontend HTML, CSS ve JavaScript ile hazırlanmış, harita entegrasyonu için Leaflet.js kütüphanesi kullanılmıştır. Backend Node.js ve Express.js ile geliştirilmiş, veritabanı olarak MySQL tercih edilmiştir. Uygulama, şehirler, gezilecek yerler ve ek içerikleri düzenli şekilde sunarak kullanıcıların interaktif harita üzerinden detaylı bilgilere ulaşmasını ve seyahat planlarını daha verimli yapmasını sağlamaktadır.

\textbf{Sonuçlar:} Proje sayesinde kullanıcılar Türkiye'nin şehirlerini kolayca keşfedip gezilecek yerler ve ilçe bilgilerine hızlıca ulaşabilmektedir. SVG tabanlı Türkiye haritası, Leaflet.js ile rota planlama ve kullanıcı dostu arayüz, seyahat planlamasını daha verimli hale getirmiştir. JWT ve bcrypt ile güvenli kullanıcı yönetimi sağlanmıştır.

\vspace{2cm}

\textbf{Anahtar Kelimeler:} Web Uygulaması, Türkiye Haritası, Seyahat Rehberi, Node.js, MySQL, Leaflet.js, JWT



\newpage
\begin{center}
\textbf{\large ABSTRACT}

\textbf{WEB-BASED TRAVEL APPLICATION \\ Traveler Map}
\end{center}

\textbf{Project Objective:} The aim of this project is to provide users who want to explore different cities in Turkey with an easy, accessible, and interactive digital guide. Through a map-based interface, users can select the city they are interested in and view places to visit, district details, and route planning features. In this way, users can both plan their trips and quickly access comprehensive information about the cities through a single platform.

\textbf{Scope of Project:} This project involves the development of a web application that will enable users to explore different cities in Turkey. The frontend is built using HTML, CSS, and JavaScript, with Leaflet.js library for map integration. The backend is developed with Node.js and Express.js, and MySQL is used as the database. The application presents cities, attractions, and additional content in an organized manner, allowing users to access detailed information through an interactive map and plan their trips more efficiently.

\textbf{Results:} Thanks to the project, users can easily explore the cities of Turkey and quickly access information about attractions and districts. The SVG-based Turkey map, Leaflet.js route planning, and user-friendly interface make travel planning more efficient. Secure user management is provided with JWT and bcrypt.

\vspace{2cm}

\textbf{Keywords:} Web Application, Turkey Map, Travel Guide, Node.js, MySQL, Leaflet.js, JWT

\pagebreak{}


