\section{SONUÇLAR VE ÖNERİLER}

Bu bölümde, gerçekleştirilen bitirme tasarım çalışmasının genel değerlendirmesi yapılmakta ve gelecekteki çalışmalar için öneriler sunulmaktadır.

\subsection{Genel Değerlendirme}

Bu çalışma kapsamında geliştirilen web tabanlı Türkiye Gezgin Haritası uygulaması, kullanıcılara Türkiye'nin farklı şehirlerini keşfetme imkanı sunan interaktif bir platform olarak başarıyla tamamlanmıştır. Uygulama, modern web teknolojileri kullanılarak geliştirilmiş ve kullanıcı dostu bir arayüz ile kullanıcılara sunulmuştur.

Projenin geliştirme sürecinde, frontend tarafında HTML5, CSS3 ve vanilla JavaScript kullanılarak responsive ve dinamik bir kullanıcı arayüzü oluşturulmuştur. SVG tabanlı interaktif Türkiye haritası, kullanıcıların illeri görsel olarak seçmesine ve ilçe detaylarına kolayca ulaşmasına olanak sağlamıştır. Backend tarafında Node.js ve Express.js framework'ü ile güçlü bir RESTful API altyapısı kurulmuş, MySQL veritabanı ile veri yönetimi gerçekleştirilmiştir.

Uygulamanın güvenlik açısından JWT tabanlı kimlik doğrulama sistemi ve bcrypt ile şifre hash'leme mekanizması kullanılması, kullanıcı verilerinin güvenliğini sağlamıştır. Ayrıca, Türkçe karakter normalizasyonu için geliştirilen özel fonksiyonlar, veritabanı sorgularında tutarlılık sağlamış ve kullanıcı deneyimini iyileştirmiştir.

Uygulama, şehirler ve ilçeler hakkında detaylı bilgiler sunarak kullanıcıların seyahat planlamasını kolaylaştırmış, gezilecek yerleri doğa, tarih ve yeme-içme kategorilerinde sunarak zengin bir içerik deneyimi sağlamıştır. JSON tabanlı veri yapısı ve MySQL veritabanının birlikte kullanılması, esneklik ve performans açısından optimum bir çözüm sunmuştur.

\subsection{Karşılaşılan Zorluklar}

Geliştirme sürecinde karşılaşılan başlıca zorluklar arasında:
\begin{itemize}
    \item Türkçe karakterlerin veritabanı ve API işlemlerinde doğru şekilde işlenmesi
    \item SVG harita üzerinde interaktif etkileşimlerin optimize edilmesi
    \item Kullanıcı kimlik doğrulama ve oturum yönetiminin güvenli şekilde gerçekleştirilmesi
    \item Farklı tarayıcı ve ekran boyutlarında tutarlı görünüm sağlanması
\end{itemize}
Bu sorunlar, çeşitli optimizasyon teknikleri ve test süreçleri ile aşılmaya çalışılmıştır.

\subsection{Gelecekteki Çalışmalar İçin Öneriler}

Bu çalışmanın ilerleyen aşamalarında:
\begin{itemize}
    \item Mobil uygulama versiyonunun geliştirilmesi (React Native veya Flutter ile)
    \item Kullanıcı yorumları ve puanlama sisteminin eklenmesi
    \item Offline çalışma desteğinin sağlanması
    \item Yapay zeka destekli kişiselleştirilmiş rota önerilerinin sunulması
    \item Sosyal medya entegrasyonu ile paylaşım özelliklerinin eklenmesi
    \item Daha kapsamlı fiyat karşılaştırma ve bütçe planlama özelliklerinin geliştirilmesi
\end{itemize}

Bu öneriler doğrultusunda yapılacak çalışmalar, geliştirilen sistemin daha verimli ve güvenilir hale getirilmesine katkı sağlayacaktır.

\subsection{Sonuç}

Gezgin Haritası projesi, web tabanlı seyahat rehberi uygulamalarına modern bir yaklaşım getirmiştir. Proje sürecinde edinilen deneyimler, web teknolojileri, veritabanı yönetimi, API tasarımı ve kullanıcı arayüzü geliştirme konularında önemli bilgi ve beceriler kazandırmıştır. Uygulamanın gelecekte yukarıda belirtilen öneriler doğrultusunda geliştirilmesi, daha kapsamlı ve kullanıcı odaklı bir platform haline gelmesini sağlayacaktır.
