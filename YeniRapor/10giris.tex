\section{GİRİŞ}
 
Dijital seyahat rehberleri, kullanıcılara farklı şehirler ve bölgeler hakkında kapsamlı bilgiler sunan bilgisayar tabanlı sistemlerdir. Bu sistemler, genellikle şehirlerin gezilecek yerleri, kullanıcı yorumları, fiyat analizleri ve diğer ek içerikler üzerinden kullanıcıya rehberlik etmeyi amaçlar. Harita tabanlı sistemler, kullanıcıların ilgilendikleri bölgeleri görsel olarak seçmelerini sağlayarak bilgiye erişimi daha kolay ve etkileşimli hale getirir. Böylece kullanıcılar, yalnızca metin tabanlı bilgiler yerine konum odaklı içeriklerle daha verimli bir deneyim yaşar.

Türkiye'nin farklı şehirlerini keşfetmek isteyen kullanıcılar için geliştirilen bu web uygulaması, seyahat planlamasını daha kolay ve erişilebilir hale getirmeyi hedeflemektedir. Uygulamanın frontend kısmı HTML, CSS ve JavaScript ile hazırlanmış, harita entegrasyonu için Leaflet.js kütüphanesi kullanılmıştır. Backend tarafında Node.js ve Express.js tercih edilmiş, veri yönetimi ve saklama işlemleri için ise MySQL veritabanı kullanılmıştır.

Uygulama, şehirler ve gezilecek yerler hakkında güncel bilgileri sunarken kullanıcı yorumları ve ortalama fiyat analizleriyle kullanıcıların karar verme süreçlerini destekler. SVG tabanlı Türkiye haritası ve Leaflet.js ile etkileşimli harita sayesinde kullanıcılar, ilgilendikleri şehri ve ilçeyi seçerek o bölgeye ait detaylı içeriklere ulaşabilirler. OpenRouteService API ile rota planlama özelliği sayesinde kullanıcılar seçtikleri yerler arasındaki mesafe ve tahmini süreyi görebilmektedir.

Sonuç olarak, proje modern web teknolojileri kullanılarak geliştirilmiş etkili bir dijital çözüm sunmakta; kullanıcıların seyahatlerini tek bir platform üzerinden planlamalarını, keşif süreçlerini kolaylaştırmalarını ve şehirler hakkında kapsamlı bilgiye hızlıca ulaşmalarını sağlamaktadır.
